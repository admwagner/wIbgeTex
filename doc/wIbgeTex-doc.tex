% !TEX TS-program = pdflatex
% ===================================================================
% LIVRO-BASE - Exemplo de uso do pacote 'wtexbase e wIbgeTex'.
% ===================================================================

% --- Classe do Documento ---
\documentclass[11pt, oneside, brazilian]{memoir}

% --- CARREGA AS CONFIGURAÇÕES DO NOSSO ARQUIVO DE ESTILO ---
\usepackage{wtexbase}
\usepackage{wIbgeTex}
\usepackage{listings}
% --- Configurações do Pacote listings ---
% --- Configurações do Pacote listings ---
\lstset{
  language=TeX,                % Informa que o código é TeX
  basicstyle=\ttfamily,        % Usa a fonte monoespaçada
  breaklines=true,             % Quebra linhas longas automaticamente
  frame=none,                  % Não adiciona moldura
  showstringspaces=false,      % Não coloca símbolos feios nos espaços
  inputencoding=utf8,
  extendedchars=true,
  literate={á}{{\'a}}1 {ã}{{\~a}}1 {à}{{\`a}}1 {â}{{\^a}}1
           {é}{{\'e}}1 {ê}{{\^e}}1
           {í}{{\'i}}1
           {ó}{{\'o}}1 {õ}{{\~o}}1 {ô}{{\^o}}1
           {ú}{{\'u}}1 {ü}{{\"u}}1
           {ç}{{\c{c}}}1
           {Á}{{\'A}}1 {Ã}{{\~A}}1 {À}{{\`A}}1 {Â}{{\^A}}1
           {É}{{\'E}}1 {Ê}{{\^E}}1
           {Í}{{\'I}}1
           {Ó}{{\'O}}1 {Õ}{{\~O}}1 {Ô}{{\^O}}1
           {Ú}{{\'U}}1 {Ü}{{\"U}}1
           {Ç}{{\c{C}}}1
           {–}{{\textendash}}1
           {º}{{\textordmasculine}}1
}

% --- Arquivos de Conteúdo Externo ---

\addbibresource{\HOMETEXW/Biblioteca/Atual/bib/2025-biblatex.bib}

% --- INFORMAÇÕES ESPECÍFICAS DESTE LIVRO ---
\title{Tentativas de Tabular IBGE}
\author{Wagner Ferreira da Silva}
\date{\the\year}



% Glossário
% \input{glossarioTeste.tex}

% Para o PDF (sobrescreve as opções padrão se necessário)
% \hypersetup{
%     pdftitle={Normas Tabular IBGE},
%     pdfauthor={Wagner Ferreira da Silva}
% }

% --- Para texto de exemplo ---
\usepackage{lipsum}


% 1. DEFINIÇÃO DO TAMANHO DO PAPEL 24 x 17
% Define tamanho físico do papel
\setstocksize{24cm}{17cm}
% Define área de composição (texto)
\settrimmedsize{24cm}{17cm}{*}
% Margens e bloco de texto
\setlrmarginsandblock{2cm}{2cm}{*} % esquerda/direita
\setulmarginsandblock{2cm}{2cm}{*} % superior/inferior
\checkandfixthelayout
 

% ===================================================================
% Início do documento
% ===================================================================
\begin{document}

\frontmatter

\begin{titlingpage}
	\centering
	\vspace*{\stretch{1}}
	{\Large \scshape \theauthor}
	\vfill
	{\Huge \bfseries \thetitle}
	\vfill
	{\large \thedate}
	\vspace*{\stretch{2}}
\end{titlingpage}

\cleardoublepage
\tableofcontents


% ---
% inserir lista de ilustrações
% ---
% \cleardoublepage
% \pdfbookmark[0]{\listfigurename}{lof}
% \addcontentsline{toc}{chapter}{Lista de figuras}
% \listoffigures*

% ---
% inserir lista de tabelas
% ---
\cleardoublepage
\pdfbookmark[0]{\listtablename}{lot}
\addcontentsline{toc}{chapter}{Lista de tabelas}
\listoftables*


\cleardoublepage



\mainmatter

\chapter{Autoria e código LaTex}


O texto e as tabelas destas anotações foram copiados ou adaptados da seguinte fonte: \fullcite{ibge-1993-tabular}.

Este documento serve a um propósito estritamente técnico. É fundamental que o leitor tenha em mente os seguintes pontos:

\begin{itemize}
    \item \textbf{Objetivo:} A finalidade exclusiva deste trabalho é desenvolver e testar o código \LaTeX{} necessário para reproduzir os modelos de tabelas do IBGE, conforme exigido em trabalhos técnicos e acadêmicos.
    
   \item  \textbf{Foco na Formatação:} O foco foi na estrutura visual e na formatação do código e não na precisão do conteúdo textual ou numérico.
    
    \item \textbf{Aviso sobre os Dados:} Consequentemente, não houve conferência da digitação, em especial dos valores apresentados nas tabelas. Os dados devem ser considerados fictícios, servindo apenas como modelos de teste, sem qualquer correspondência com a realidade.

\end{itemize}
 
\chapter{Guia de Utilização do Pacote \texttt{wIbgeTex}}

Este capítulo serve como manual de utilização do pacote \texttt{wIbgeTex.sty}, desenvolvido a partir dos exemplos do manual do IBGE.

Nos exemplos, o código \texttt{\string,} (\textit{thin space}) gera um pequeno espaço entre os números, substituindo o ponto utilizado como separador de milhar. Em alguns casos, a vírgula foi colocada entre chaves (como em \texttt{\{,\}}), para evitar espaço desnecessário.


\section{Carregamento e Dependências}

Para formatar tabelas no padrão IBGE, basta carregar o pacote \texttt{wIbgeTex} no preâmbulo do seu documento, após a classe utilizada (ex.: \texttt{memoir} ou \texttt{abntex2}).

\begin{quadroBase}[Carregamento do Pacote (Preâmbulo)]
{\footnotesize
\begin{verbatim}
% Carrega o pacote (ex: \usepackage{wIbgeTex} ou \usepackage{../wIbgeTex})
\usepackage{wIbgeTex}
\end{verbatim}
}
\end{quadroBase}

O pacote \texttt{wIbgeTex.sty} carrega automaticamente todas as dependências necessárias para o usuário, incluindo:

\begin{itemize}
\item \texttt{ctable} e \texttt{booktabs} (para a construção das molduras das tabelas);
\item \texttt{tabularx}, \texttt{longtable} e \texttt{xltabular} (para tabelas com largura flexível ou que se estendem por mais de uma página);
\item \texttt{siunitx} (para formatação de números e ângulos);
\item \texttt{tikz} (para símbolos gráficos);
\item \texttt{caption} (para formatação das legendas);
\item \texttt{xparse} e \texttt{etoolbox} (para comandos personalizados).
\end{itemize}

\textbf{Nota importante:} O pacote \textbf{não} carrega as fontes principais do documento. É necessário carregar, no preâmbulo, as fontes de sua escolha (como \texttt{XCharter} e \texttt{tgheros}). Por exemplo, o ajuste de fontes pode ser feito com:

\begin{quadroBase}[Exemplo de pacotes para fontes (Preâmbulo)]
{\footnotesize
	\begin{verbatim}
\usepackage{XCharter}             % fonte principal Charter 
\usepackage[condensed]{tgheros}   % Fonte sans-serif 
\usepackage{newpxmath}            % Matemática compatível
	\end{verbatim}
}
\end{quadroBase}

\section{Formatação Automática}

Quando carregado, o pacote \texttt{wIbgeTex.sty} aplica automaticamente as seguintes regras de formatação do IBGE:

\begin{itemize}
\item \textbf{Fonte da Tabela:} aplica a fonte definida em \texttt{\string\wIbgeTabFont} (por padrão, \texttt{\string\sffamily\string\footnotesize}) a todos os ambientes \texttt{tabular}, \texttt{tabularx}, \texttt{longtable} e \texttt{xltabular};
\item \textbf{Legendas (Caption):} configura as legendas para o padrão IBGE, alinhadas à esquerda e com travessão (ex.: \textbf{Tabela 1 \textendash{} Título});
\item \textbf{Espaçamento de Linha:} ajusta o espaçamento vertical (\texttt{\string\extrarowheight} e \texttt{\string\arraystretch}) para proporcionar maior legibilidade;
\item \textbf{Bordas da Tabela:} remove os espaços verticais adicionais do \texttt{booktabs} (\texttt{\string\aboverulesep} e \texttt{\string\belowrulesep}), garantindo linhas mais próximas ao texto.
\end{itemize}


\section{Comandos Disponíveis}

O pacote fornece um conjunto de comandos com o prefixo \texttt{wIbge} para padronizar a criação das tabelas pelo usuário.

\subsection{Tipos de Coluna}

O pacote disponibiliza os seguintes tipos de coluna personalizados para uso nos ambientes \texttt{ctable}, \texttt{tabularx}, \texttt{tabular}, \texttt{longtable} e \texttt{xltabular}:

\begin{itemize}
    \item \textbf{Alinhados pela Base (tipo \texttt{b})}
        \begin{itemize}
            \item \texttt{L\{<largura>\}}: Coluna alinhada à esquerda (base);
            \item \texttt{C\{<largura>\}}: Coluna centralizada (base);
            \item \texttt{R\{<largura>\}}: Coluna alinhada à direita (base);
        \end{itemize}
        
    \item \textbf{Alinhados ao Meio (tipo \texttt{m})}
        \begin{itemize}
            \item \texttt{E\{<largura>\}}: Coluna alinhada à esquerda (meio);
            \item \texttt{M\{<largura>\}}: Coluna centralizada (meio);
            \item \texttt{D\{<largura>\}}: Coluna alinhada à direita (meio);
        \end{itemize}
        
    \item \textbf{Coluna Expansível (tipo \texttt{X})}
         \begin{itemize}
            \item \texttt{Y}: Uma coluna \texttt{X} (expansível) que alinha o texto à direita.
         \end{itemize}
\end{itemize}


\subsection{Fontes e Notas de Rodapé}

Estes comandos devem ser utilizados:
\begin{itemize}
    \item no bloco de notas (\texttt{\{\% notas ... \}}) do \texttt{ctable}, como ocorre na maior parte dos exemplos apresentados;
    \item entre \texttt{\string\end{tabularx}} e \texttt{\string\end{table}}, como na \wautoref{tabIbge2};
    \item ou após \texttt{\string\end{longtable}} ou \texttt{\string\end{xltabular}}, como na \wautoref{tabIbge3}.
\end{itemize}

\begin{quadroBase}[Comandos \texttt{\string\wIbgeFonte} e \texttt{\string\wIbgeNota}]
{\footnotesize
\begin{verbatim}
{% notas
  % --- Comando para a Fonte ---
  % O rótulo ``Fonte:'' é o padrão.
  \wIbgeFonte{Fundação Instituto Brasileiro de Geografia e 
             Estatística (IBGE)}

  % --- Comando para Notas ---
  % O rótulo ``Nota:'' é o padrão.
  \wIbgeNota{Esta é uma nota geral.}
  
  % --- Para rótulos personalizados (ex: ``Fontes'' ou ``Notas'') ---
  \wIbgeFonte[Fontes]{IBGE e Ministério X.}
  \wIbgeNota[Notas]{Esta é a nota 1. \par Esta é a nota 2.}
}
\end{verbatim}
}
\end{quadroBase}

Observe que, em todos os exemplos do manual do IBGE, as tabelas ocupam toda a largura disponível. Assim, ambos os comandos foram programados para garantir o alinhamento perfeito à esquerda, mesmo quando os rótulos (Fonte, Fontes, Nota, Notas) apresentam larguras diferentes. Esse aspecto pode ser observado na \wautoref{tabIbge1}, que, em tese, poderia ter largura menor.



\subsection{Sinais Convencionais}\label{secaoComandoSinalConvencional}
Para padronizar a lista de sinais convencionais, utilize o comando \texttt{\string\wIbgeSinaisConvencionais}. Ele aceita uma lista de chaves (os próprios símbolos) separadas por vírgula e trata automaticamente da pluralização do título (“Sinal” ou “Sinais”).

\begin{quadroBase}[Comando \texttt{\string\wIbgeSinaisConvencionais}]
{\footnotesize
\begin{verbatim}
% Exemplo para Tabela 14:
\wIbgeNota{
  \wIbgeSinaisConvencionais{x, -}
}

% Exemplo para Tabela 3 (com outros textos):
\wIbgeNota[Notas]{Dados numéricos arredondados. \par 
  \wIbgeSinaisConvencionais{0.00, -0.00}
}
\end{verbatim}
}
\end{quadroBase}

As chaves disponíveis e as suas descrições correspondentes (conforme definido no \texttt{wIbgeTex.sty}) são:

\noindent
\begin{tabularx}{\linewidth}{lX}
	\toprule
	\textbf{Chave} & \textbf{Descrição gerada}                                                      \\
	\midrule
	\textbf{-}     & Dado numérico igual a zero não resultante de arredondamento;              \\
	\textbf{..}    & Dado numérico não se aplica;                                              \\
	\textbf{...}   & Dado numérico não disponível;                                             \\
	\textbf{x}     & Dado numérico omitido a fim de evitar a individualização da informação;   \\[4pt]

  \makecell[l]{%
    \begin{tabular}{@{}l@{\hspace{1em}}r@{}}
      \textbf{0}    & \wIbgeTikzmark{A1} \\
      \textbf{0,0}  & \\
      \textbf{0,00} & \wIbgeTikzmark{A2}
    \end{tabular}%
  } &
  Dado numérico igual a zero resultante do arredondamento de um dado originalmente positivo; \\[6pt]

  \makecell[l]{%
    \begin{tabular}{@{}l@{\hspace{1em}}r@{}}
      \textbf{-0}    & \wIbgeTikzmark{B1} \\
      \textbf{-0,0}  & \\
      \textbf{-0,00} & \wIbgeTikzmark{B2}
    \end{tabular}%
  } &
  Dado numérico igual a zero resultante do arredondamento de um dado originalmente negativo. \\
  \bottomrule
\end{tabularx}

% desenha as chaves (após a tabela)
\wIbgeDrawbrace{A1}{A2}
\wIbgeDrawbrace{B1}{B2}


\subsection{Símbolos e Coordenadas}

\begin{itemize}
  \item \textbf{Símbolos de intervalo:} para uso em texto corrente, o pacote fornece os comandos \texttt{\string\wIbgeVdash} ($w \wIbgeVdash z$), \texttt{\string\wIbgeDashv} ($w \wIbgeDashv z$) e \texttt{\string\wIbgeVdashDashv} ($w \wIbgeVdashDashv z$);
  
\item \textbf{Coordenadas:} o pacote carrega e configura o \texttt{siunitx}. Use o comando \texttt{\string\ang{}} para formatar coordenadas. O \texttt{siunitx} detecta automaticamente a fonte utilizada (\texttt{\string\wIbgeTabFont}) na tabela.

\end{itemize}

\begin{quadroBase}[Uso de \texttt{\string\ang} (siunitx)]
{\footnotesize
\begin{verbatim}
% No seu .tex:
Pico da Neblina & ... & \ang{+00;47;49} & \ang{-66;00;22} \\

% Resultado (em fonte sans-serif pequena):
+00°47'49"   -66°00'22"
\end{verbatim}
}
\end{quadroBase}


\section{Customização Avançada}


Se necessário alterar a fonte padrão das tabelas, é possível redefinir os seguintes comandos no preâmbulo do documento, \textbf{após} o carregamento do \texttt{wIbgeTex.sty}:


\begin{quadroBase}[Customização de Fontes (Preâmbulo)]
{\footnotesize
\begin{verbatim}
\usepackage{wIbgeTex}

% Exemplo: Mudar a fonte de todas as tabelas para \small
\renewcommand{\wIbgeTabFont}{\sffamily\small}
\renewcommand{\wIbgeNoteFont}{\sffamily\small}
\end{verbatim}
}
\end{quadroBase}


\chapter{Princípios de Elaboração de Tabelas (IBGE)}

Este capítulo resume os principais componentes de uma tabela, conforme as normas do IBGE \cite{ibge-1993-tabular}.



\section{Numeração}
Se um documento contiver duas ou mais tabelas, estas devem ser numeradas sequencialmente.

\begin{itemize}
    \item Use algarismos arábicos 
    (ex: \textbf{Tabela 1} e \textbf{Tabela 2}).
    \item A numeração pode ser contínua ou vinculada ao capítulo 
    (ex: \textbf{Tabela 1.1}).
\item A referência a esta regra pode ser encontrada na ABNT NBR 14724.
\end{itemize}

Vide modelo em \wautoreftabappendix{tabIbge1}{anexoTabelas}.


\section{Título}
O título é obrigatório e deve ser posicionado no topo da tabela. Sua função é descrever o \textbf{quê} (natureza), \textbf{onde} (abrangência geográfica) e \textbf{quando} (abrangência temporal) os dados foram observados.


\begin{itemize}
    \item \textbf{Clareza:} O título deve ser claro, conciso e escrito por extenso, sem abreviações.
    \item \textbf{Natureza dos Dados:} Se a tabela contiver apenas números absolutos, é dispensável indicar ``em números absolutos'' no título.
    \item \textbf{Abrangência Temporal:} Deve ser clara, indicando o ponto no tempo ou o período da série. Consulte o \wautoref{cap5IBGE} para detalhes.
\end{itemize}


\textbf{Exemplos:}

\begin{itemize}
	\setlength{\labelsep}{0pt}
	\renewcommand{\labelitemi}{}
	\item Produção acumulada de casulos do bicho-da-seda dos estabelecimentos, por Unidade da Federação, no período 1980--1990, Brasil;
	\item Produção de casulos do bicho-da-seda, em números absolutos e relativos, por Unidade da Federação -- Brasil -- 1974;
	\item Produção média de casulos do bicho-da-seda dos estabelecimentos, por Unidade da Federação -- Brasil -- 1974;
	\item Pessoas empregadas em atividades agrícolas, por grupo de horas semanais trabalhadas e classe de remuneração mensal -- Brasil -- 1976;

	\item Números-índices mensais de preços mínimos, acumulados em 12 meses, de frutas cítricas da Região Metropolitana de Curitiba -- 1990--1991;
	\item Mortes registradas de 1956 a 1964 que ocorreram entre 1951 e 1964, Japão;

	\item Produção de casulos do bicho-da-seda, por Unidade da Federação -- Brasil -- 1974.
\end{itemize}


\subsection{Moldura}
A moldura define a estrutura da tabela.
\begin{itemize}
    \item A tabela não deve ter linhas verticais externas (à esquerda ou à direita).
    \item Deve conter, no mínimo, três linhas horizontais: uma no topo (acima do cabeçalho), uma abaixo do cabeçalho e uma no rodapé (fechando a tabela).
    \item Linhas verticais ou horizontais internas (\texttt{\string\cmidrule} ou \texttt{|}) podem ser empregadas para destacar ou separar informações complexas no cabeçalho ou no corpo.
\end{itemize}


\textbf{Exemplo:} \autorefanexo{anexoTabelas}, \wautorefLista{tabIbge3,tabIbge5,tabIbge6,tabIbge7,tabIbge9,tabIbge10,tabIbge11,tabIbge12,tabIbge14,tabIbge15}. 


\subsection{Cabeçalho}
O cabeçalho (topo da tabela) especifica o conteúdo de cada coluna, complementando as informações do título. Deve ser claro e, preferencialmente, escrito por extenso.

\subsection{Indicador de Linha}

Refere-se ao conteúdo da primeira coluna (ou colunas) da tabela, que descreve cada linha da tabela.


\subsection{Unidade de Medida}
A unidade de medida deve ser indicada (geralmente entre parênteses) no cabeçalho ou no indicador de linha.


\textbf{Exemplo:} \autorefanexo{anexoTabelas},
\wautorefLista{tabIbge3,tabIbge4,tabIbge7,tabIbge8,tabIbge9,tabIbge10,tabIbge12,tabIbge13,tabIbge14,tabIbge15}.


\textbf{Exemplos:}
(m) ou (metro); (t) ou (tonelada); (R\$) ou (real); (t/km) ou (toneladas por quilômetro); (hab/km\textsuperscript{2}) ou (habitantes por quilômetro quadrado).




\textbf{Exemplos:}

\begin{itemize}
	\setlength{\labelsep}{0pt}
	\renewcommand{\labelitemi}{}
	\item (1~000~t) ou (1000~t) --- indica dados numéricos em toneladas que foram divididos por mil;
	\item (1~000~R\$) ou (1000~R\$) --- indica dados numéricos em reais que foram divididos por mil;
	\item (\%) ou (percentual) --- indica dados numéricos proporcionais a cem;
	\item (\textperthousand) ou (por mil) --- indica dados numéricos proporcionais a mil;
	\item (1/1000) --- indica dados numéricos que foram divididos por 1/1000, ou seja, multiplicados por mil.
\end{itemize}



\subsection{Dado numérico}

Os dados devem ser apresentados em algarismos arábicos. A grafia (ex.: o uso da vírgula como separador decimal) deve seguir as normas do CONMETRO.



\subsection{Sinal convencional}\label{secaoSinalConvecional}


Sinais são usados para substituir dados numéricos ausentes ou indisponíveis. Quando empregados, devem ser explicados em nota geral. O pacote \texttt{wIbgeTex} fornece o comando \texttt{\string\wIbgeSinaisConvencionais} para padronizar essa explicação (consulte as \wautoref{tabGuia-sinais-conv} e \wautoref{secaoComandoSinalConvencional}).



\noindent\begin{table}[htb]
\caption{Sinais convencionais e seus significados}\label{tabGuia-sinais-conv}
\begin{tabularx}{\linewidth}{lX}
	\toprule
	\textbf{Sinal} & \textbf{Significado}                                                      \\
	\midrule
	\textbf{--}     & Dado numérico igual a zero não resultante de arredondamento;              \\
	\textbf{..}    & Dado numérico não se aplica;                                              \\
	\textbf{...}   & Dado numérico não disponível;                                             \\
	\textbf{x}     & Dado numérico omitido a fim de evitar a individualização da informação;   \\[4pt]

    \makecell[l]{%
    \begin{tabular}{@{}l@{\hspace{1em}}r@{}}
      \textbf{0}    & \wIbgeTikzmark{A1} \\
      \textbf{0,0}  & \\
      \textbf{0,00} & \\
      \textbf{etc.} & \wIbgeTikzmark{A2}
    \end{tabular}%
  } &
  Dado numérico igual a zero resultante do arredondamento de um dado originalmente positivo; \\[6pt]

  \makecell[l]{%
    \begin{tabular}{@{}l@{\hspace{1em}}r@{}}
      \textbf{-0}    & \wIbgeTikzmark{B1} \\
      \textbf{-0,0}  & \\
      \textbf{-0,00} & \\
      \textbf{etc.} & \wIbgeTikzmark{B2}
    \end{tabular}%
  } &
  Dado numérico igual a zero resultante do arredondamento de um dado originalmente negativo. \\
  \bottomrule
\end{tabularx}
\end{table}

% desenha as chaves (após a tabela)
\wIbgeDrawbrace{A1}{A2}
\wIbgeDrawbrace{B1}{B2}




\textbf{Exemplo:}\autorefanexo{anexoTabelas},
\wautorefLista{tabIbge2,tabIbge3,tabIbge4,tabIbge10,tabIbge12,tabIbge13,tabIbge14}.

\textit{Nota:} No caso de publicação que contenha tabelas com sinais convencionais, em que a apresentação dos sinais e de seus significados figure em destaque, é dispensável a nota geral em cada tabela.


\subsection{Chamada, Fonte e Notas}
\begin{itemize}
    \item \textbf{Chamada:} É um símbolo (normalmente um número entre parênteses) usado no corpo da tabela (com \texttt{\string\tmark}) (do pacote \texttt{ctable}) para remeter a uma nota específica.
\item \textbf{Nota Específica:} Esclarece um ponto específico marcado por uma ``chamada''. Use o comando \texttt{\string\tnote} (do pacote \texttt{ctable}).

    \item \textbf{Fonte:} Indica a origem dos dados. É um elemento obrigatório e deve ser inserido no rodapé da tabela. Use o comando \texttt{\string\wIbgeFonte}.
    \item \textbf{Nota Geral:} Esclarece o conteúdo geral da tabela ou o uso de sinais. Use o comando \texttt{\string\wIbgeNota}.
   
\end{itemize}

\textbf{Exemplos de Notas:}\autorefanexo{anexoTabelas},
\wautorefLista{tabIbge2,tabIbge3,tabIbge4,tabIbge5,tabIbge8,tabIbge10,tabIbge11,tabIbge12,tabIbge13,tabIbge14,tabIbge15}.


\chapter{Apresentação de Tempo e Períodos}\label{cap5IBGE}

A formatação de datas e períodos deve ser precisa e desprovida de ambiguidades.

\begin{itemize}
    \item \textbf{Séries consecutivas:} utilize hífen (ou travessão \textendash{}) para ligar o ponto inicial ao ponto final. Exemplo: \textbf{1981--1985} (inclui todos os anos de 1981 a 1985).
    \item \textbf{Séries não consecutivas:} utilize barra (/) para delimitar os pontos. Exemplo: \textbf{1981/1985} (inclui apenas os anos de 1981 e 1985). Quando houver poucos pontos, podem ser usadas vírgulas, como em \textbf{1988, 1990, 1991}.
    \item \textbf{Safras:} utilize barra entre os dois últimos algarismos dos anos. Exemplo: \textbf{Safra 91/92}.
\end{itemize}

Quando a tabela contiver dados numéricos referentes a um período anual distinto do ano civil, essa informação deve ser indicada no título, em nota geral ou em nota específica.



\chapter{Apresentação de classe de frequência}



Classes de frequência (intervalos numéricos) devem ser apresentadas de forma não ambígua. O pacote \texttt{wIbgeTex} fornece comandos para os símbolos de intervalo padrão:

\begin{itemize}
    \item \textbf{Inclui o início, exclui o fim} ($w$ a menos de $z$): Utilize \texttt{\string\wIbgeVdash}. Ex.: $w \wIbgeVdash z$.
    \item \textbf{Exclui o início, inclui o fim} (mais de $w$ a $z$): Utilize \texttt{\string\wIbgeDashv}. Ex.: $w \wIbgeDashv z$.
    \item \textbf{Inclui ambos os extremos} ($w$ a $z$): Utilize \texttt{\string\wIbgeVdashDashv}. Ex.: $w \wIbgeVdashDashv z$.
\end{itemize}

Recomenda-se evitar classes abertas (como ``menos de $z$'' ou ``$w$ ou mais'') e adotar classes inicial e final fechadas na distribuição de frequência. Devem-se evitar expressões como “até~$z$”, “menos de~$z$”, “$w$~ou~mais” e “mais de~$w$”.

\textbf{Exemplo:} \autorefanexo{anexoTabelas}, \wautorefLista{tabIbge8,tabIbge9,tabIbge10,tabIbge12}.


\chapter{Arredondamento de dado numérico}

Se os dados forem arredondados, essa informação deve ser indicada em nota.

O arredondamento dos dados numéricos deve respeitar as diferenças significativas, absolutas e relativas, existentes entre eles.

\textbf{Exemplo:} \autorefanexo{anexoTabelas}, \wautorefLista{tabIbge3}.

 
Quando, em uma tabela, após o arredondamento dos dados numéricos, houver divergência entre a soma das parcelas arredondadas e o total arredondado, deve-se adotar um dos seguintes procedimentos:

\begin{enumerate}[label=\alph*)]
	\item Inclusão de uma nota geral esclarecendo a divergência;

	\item Correção na parcela (ou parcelas) cujo valor absoluto da razão entre a diferença de arredondamento (dado numérico original menos dado corrigido) e o dado numérico original seja menor.
\end{enumerate}

\textbf{Exemplo:}


\begin{center}
	\begin{tabular}{ccc}
		\toprule
		\textbf{Dado numérico original} & \textbf{Dado numérico arredondado} \\
		\midrule
		7{,}6                           & 8                                  \\
		11{,}6                          & 12                                 \\
		20{,}2                          & 20                                 \\
		\midrule
		\textbf{Total}                  & \textbf{39}                        \\
		\bottomrule
	\end{tabular}
\end{center}

\vspace{1em}

Porém, a soma das parcelas arredondadas resulta em:
\[
	8 + 12 + 20 = 40
\]

\textbf{Soluções possíveis:}


\begin{center}
	\begin{tabular}{ccc}
		\toprule
		\textbf{Solução} & \textbf{Parcela ajustada} & \textbf{Total corrigido} \\
		\midrule
		1                & $7 + 12 + 20$             & 39                       \\
		2                & $8 + 11 + 20$             & 39                       \\
		3                & $8 + 12 + 19$             & 39                       \\
		\bottomrule
	\end{tabular}
\end{center}


\textbf{Cálculo da razão:}

\[
	\begin{aligned}
		(7{,}6 - 7) / 7{,}6    & = 0{,}079; \\
		(11{,}6 - 11) / 11{,}6 & = 0{,}052; \\
		(20{,}2 - 19) / 20{,}2 & = 0{,}059.
	\end{aligned}
\]

Como $0{,}052 < 0{,}059 < 0{,}079$, a solução recomendada é a \textbf{nº 2}, isto é:


\begin{center}
	\begin{tabular}{cc}
		\toprule
		\textbf{Dado numérico corrigido} & \textbf{Valor adotado} \\
		\midrule
		7{,}6                            & 8                      \\
		11{,}6                           & 11                     \\
		20{,}2                           & 20                     \\
		\midrule
		\textbf{Total}                   & 39                     \\
		\bottomrule
	\end{tabular}
\end{center}



Quando, em uma tabela, após o arredondamento de um dado numérico, o resultado for $0$, $0{,}0$, $0{,}00$ e assim por diante, o resultado deve ser apresentado como $0$ ou $-0$; $0{,}0$ ou $-0{,}0$; $0{,}00$ ou $-0{,}00$, conservando-se o sinal do dado numérico original. Essa distinção evita confusão com o dado numérico igual a zero, representado por outro sinal convencional conforme a \wautoref{secaoSinalConvecional}.

\textbf{Exemplo:} \autorefanexo{anexoTabelas}, \wautorefLista{tabIbge3}.


\chapter{Diagramação e Recomendações Gerais}

\begin{itemize}
   
   \item  \textbf{Quebra de tabelas:} se uma tabela for extensa, ela deve ser dividida. O cabeçalho deve ser repetido em cada nova página. O \texttt{longtable} ou o \texttt{xltabular}  (usado na \wautoref{tabIbge3}) realiza essa operação automaticamente.

    
    \item \textbf{Quebra horizontal:} tabelas com muitas colunas e poucas linhas podem ser divididas em blocos, um abaixo do outro, repetindo-se os indicadores de linha (ex.: \wautorefLista{tabIbge2,tabIbge4}).
    
    \item \textbf{Quebra lado a lado:} tabelas com poucas colunas e muitas linhas podem ser diagramadas lado a lado na mesma página (ex.: \wautorefLista{tabIbge3,tabIbge6}).
    
    \item \textbf{Clareza:} procure fazer a tabela caber em uma única página. Evite o excesso de células com sinais convencionais, preferindo dados numéricos, e mantenha a uniformidade gráfica entre todas as tabelas.
   
\end{itemize}


\backmatter  % parte não textual



%=========================================================
% ANEXOS
%=========================================================
\begin{anexosenv}
	\chapter{Exemplos de tabelas}\label{anexoTabelas}


  % --- Importa a Tabela 1 ---
	% Tabela Modelo 1
% Extraída ou adaptada de:
% IBGE, Fundação Instituto Brasileiro de Geografia e Estatística.
% Normas de apresentação tabular. 3. ed. Rio de Janeiro: IBGE, 1993.
%
% Este arquivo é destinado a ser incluído via \input{}

\ctable[
  caption = {Pessoas residentes em domicílios particulares, por sexo e 
    situação do domicílio -- Brasil -- 1980}, 
  label = tabIbge1,
  width = \linewidth, center, pos = !htb, notespar, nosuper
]
{X R{25mm} R{25mm} R{25mm}} % Definição das colunas
{% notas
  \wIbgeFonte{Fundação Instituto Brasileiro de Geografia e Estatística (IBGE)}
}{ % conteúdo da tabela 
  \FL
  Situação do domicílio & Total & Mulheres & Homens \ML
  \hspace{10mm} Total & 177960301 & 59595332 & 58364969 \\
  \addlinespace\addlinespace
  Urbana & 79972931 & 41115439 & 38857492 \\
  \addlinespace
  Rural & 37987370 & 18479893 & 19507477 \LL
}


  \begin{quadroBase}[Código Tabela 1]
  {\footnotesize
    \lstinputlisting[firstline=8]{../exemplo/tabelas/tabela-01-modelo.tex} 
  }
  \end{quadroBase}
 
% --- Importa a Tabela 2 ---
\clearpage
% Tabela Modelo 2
% Extraída ou adaptada de:
% IBGE, Fundação Instituto Brasileiro de Geografia e Estatística.
% Normas de apresentação tabular. 3. ed. Rio de Janeiro: IBGE, 1993.
%
% Este arquivo é destinado a ser incluído via \input{}

% Modelo 2 usa 'table' e 'tabularx' porque o ctable não suporta
% a quebra da tabela em duas partes horizontais.
\begin{table}[htb]
  \caption{Pessoas residentes em domicílios particulares, por estado 
    conjugal, para as Microrregiões e os Municípios do Estado do 
    Amapá -- 1980} 
  \label{tabIbge2}
  
  %%%%% Quadro 1 (Parte 1 da tabela) 
  \begin{tabularx}{\linewidth}{L{50mm} Y Y Y Y}
    \toprule \addlinespace
    & \textbf{Total} & \textbf{Solteiro} & \textbf{Casado} & 
      \textbf{Separado} \\ \addlinespace    \midrule
    \hspace{10mm} Total \dotfill & 89\,264 & 30\,509 & 51\,327 & 2\,412 \\
    \addlinespace
    \multicolumn{5}{l}{\hspace{5mm} Microrregiões} \\
    Macapá \dotfill & 80\,920 & 28\,012 & 46\,042 & 2\,288 \\
    Amapá e Oiapoque \dotfill & 8\,344 & 2\,497 & 5\,285 & 124 \\
    \addlinespace
    \multicolumn{5}{l}{\hspace{5mm} Municípios} \\
    Amapá \dotfill & 4\,551 & 1\,405 & 2\,844 & 61 \\
    Calçoene \dotfill & 1\,352 & 474 & 770 & 39 \\
    Macapá \dotfill & 70\,829 & 25\,168 & 39\,502 & 2\,034 \\
    Mazagão \dotfill & 10\,091 & 2\,844 & 6\,540 & 254 \\
    Oiapoque \dotfill & 2\,441 & 618 & 1\,671 & 24 \\ 
    \bottomrule
  \end{tabularx}
  
  \par \vspace{0.5em} % Pequeno espaço entre as duas partes

  %%%%% Quadro 2 (Parte 2 da tabela) 
  \begin{tabularx}{\linewidth}{L{50mm} Y Y Y}
    \toprule \addlinespace
    & \textbf{Desquitado} & \textbf{Viúvo} & \textbf{Sem declaração} \\ 
    \addlinespace \midrule
    \hspace{10mm} Total \dotfill & 152 & 3\,762 & 1\,102 \\
    \addlinespace
    \multicolumn{4}{l}{\hspace{5mm} Microrregiões} \\
    Macapá \dotfill & 152 & 3\,406 & 1\,020 \\
    Amapá e Oiapoque \dotfill     & - & 356 & 82 \\
    \addlinespace
    \multicolumn{4}{l}{\hspace{5mm} Municípios} \\
    Amapá \dotfill & - & 189 & 52 \\
    Calçoene \dotfill & - & 66 & 3 \\
    Macapá \dotfill & 128 & 3\,080 & 917 \\
    Mazagão \dotfill & 24 & 326 & 103 \\
    Oiapoque \dotfill & - & 101 & 27 \\
    \bottomrule
  \end{tabularx}
  
  \wIbgeFonte{Fundação Instituto Brasileiro de Geografia e Estatística (IBGE)}
  \wIbgeNota{\wIbgeSinaisConvencionais{-}
  }

\end{table}

\newpage
\begin{quadroBase}[Código Tabela 2]
{\footnotesize
  \lstinputlisting[firstline=8]{../exemplo/tabelas/tabela-02-modelo.tex} 
}
\end{quadroBase}

% --- Importa a Tabela 3 ---
\clearpage
% Tabela Modelo 3
% Extraída ou adaptada de:
% IBGE, Fundação Instituto Brasileiro de Geografia e Estatística.
% Normas de apresentação tabular. 3. ed. Rio de Janeiro: IBGE, 1993.
%
% Este arquivo é destinado a ser incluído via \input{}

{\renewcommand{\arraystretch}{1.1}\setlength{\LTcapwidth}{\linewidth}
\begin{SingleSpacing}
% zera o \parskip pois a fonte e notas estão após
% o longtable como parágrafos normais
\setlength{\parskip}{0pt}

\begin{longtable}{L{35mm} R{20mm} || L{35mm} R{20mm}}
  
  \caption{Taxa de crescimento anual da população residente, em ordem 
    decrescente, por Municípios do Estado de Alagoas, no período 
      1980--1991}
  \label{tabIbge3}\\

  \toprule
  \multicolumn{1}{C{35mm}|}{Municípios} & 
    \multicolumn{1}{C{20mm}||}{Taxa de crescimento anual (\%)} &
  \multicolumn{1}{C{35mm}|}{Municípios} & 
    \multicolumn{1}{C{20mm}}{Taxa de crescimento anual (\%)} \\
  \midrule
\endfirsthead

  % Cabeçalho para as páginas seguintes
  \caption{Taxa de crescimento anual da população residente, em ordem 
    decrescente, por Municípios do Estado de Alagoas, no período 
    1980--1991} \\
  \multicolumn{4}{r}{(continuação)} \\
  \toprule
  \multicolumn{1}{C{35mm}|}{Municípios} & 
    \multicolumn{1}{C{20mm}||}{Taxa de crescimento anual (\%)} &
  \multicolumn{1}{C{35mm}|}{Municípios} & 
    \multicolumn{1}{C{20mm}}{Taxa de crescimento anual (\%)} \\
  \midrule
\endhead

  % Rodapé para páginas intermediárias
  \multicolumn{4}{r}{\textit{Continua \ldots}}\\
\endfoot

  % Rodapé final (na última página)
  \bottomrule
\endlastfoot

  % Corpo da tabela:
  Piranhas & 8{,}44 & Penedo & 3{,}26 \\
  Campo Alegre & 7{,}07 & Messias & 3{,}19 \\
  Barra de São Miguel & 7{,}05 & Cajueiro & 3{,}03 \\
  Santa Luzia do Norte & 5{,}28 & Jaramataia & 2{,}99 \\
  Japaratinga & 4{,}83 & Joaquim Gomes & 2{,}74 \\
  Teotônio Vilela & 4{,}42 & Arapiraca & 2{,}61 \\
  Maceió & 4{,}21 & Coruripe & 2{,}57 \\
  Olho d'Água do Casado & 4{,}14 & Cacimbinhas & 2{,}38 \\
  Delmiro Gouveia & 4{,}00 & Ibateguara & 2{,}36 \\
  Craíbas & 3{,}87 & Feliz Deserto & 2{,}26 \\
  Barra de Santo Antônio & 3{,}61 & Junqueiro & 2{,}25 \\
  Satuba & 3{,}60 & Taquarana & 2{,}17 \\
  Piaçabuçu & 3{,}59 & Lagoa da Canoa & 2{,}12 \\
  Palestina & 3{,}52 & Dois Riachos & 2{,}11 \\
  Roteiro & 3{,}50 & Coqueiro Seco & 2{,}10 \\
  Jundiá & 3{,}29 & Batalha & 2{,}08 \\
  São Sebastião & 2{,}03 & Pão de Açúcar & 1{,}17 \\
  Passo de Camaragibe & 1{,}99 & Minador do Negrão & 1{,}14 \\
  São Miguel dos Campos & 1{,}99 & Monteirópolis & 1{,}13 \\
  Girau do Ponciano & 1{,}97 & Mata Grande & 1{,}08 \\
  Belo Monte & 1{,}98 & Olho d'Água das Flores & 1{,}06 \\
  Rio Largo & 1{,}96 & Colônia Leopoldina & 1{,}03 \\
  Matriz de Camaragibe & 1{,}91 & Murici & 0{,}97 \\
  Jacaré dos Homens & 1{,}86 & Santana do Ipanema & 0{,}95 \\
  Pilar & 1{,}83 & Porto Calvo & 0{,}94 \\
  Boca da Mata & 1{,}88 & São José da Tapera & 0{,}98 \\
  Porto Real do Colégio & 1{,}80 & Anadia & 0{,}88 \\
  São Luís do Quitunde & 1{,}70 & Maragogi & 0{,}83 \\
  Senador Rui Palmeira & 1{,}68 & Coité do Noia & 0{,}81 \\
  Traipu & 1{,}46 & União dos Palmares & 0{,}79 \\
  Palmeira dos Índios & 1{,}29 & Feira Grande & 0{,}75 \\
  Inhapi & 1{,}28 & Major Isidoro & 0{,}71 \\
  Campo Grande & 0{,}70 & Maribondo & -0{,}08 \\
  Poço das Trincheiras & 0{,}67 & Porto de Pedras & -0{,}12 \\
  Marechal Deodoro & 0{,}60 & Maravilha & -0{,}38 \\
  Limoeiro de Anadia & 0{,}59 & Viçosa & -0{,}40 \\
  Ouro Branco & 0{,}57 & Olho d'Água Grande & -0{,}42 \\
  Olivença & 0{,}55 & Mar Vermelho & -0{,}45 \\
  Igaci & 0{,}55 & Belém & -0{,}48 \\
  Água Branca & 0{,}49 & Atalaia & 0{,}72 \\
  Carneiros & 0{,}39 & Quebrângulo & -0{,}93 \\
  Igreja Nova & 0{,}34 & Santana do Mundaú & -1{,}18 \\
  Tanque d'Arca & 0{,}24 & Branquinha & -1{,}25 \\
  São Miguel dos Milagres & 0{,}16 & Paulo Jacinto & -1{,}27 \\
  Canapi & 0{,}09 & Flexeiras & -1{,}33 \\
  Capela & 0{,}08 & São Brás & -1{,}36 \\
  São José da Laje & 0{,}00 & Chã Preta & -1{,}67 \\
  Jacuípe & -0{,}00 & Pindoba & -2{,}93 \\
  Novo Lino & -0{,}06 & & \\
\end{longtable}

\wIbgeFonte{Fundação Instituto Brasileiro de Geografia e Estatística (IBGE)}
\wIbgeNota[Notas]{Dados numéricos arredondados. \par 
    \wIbgeSinaisConvencionais{0.00, -0.00}
}

\end{SingleSpacing}
\renewcommand{\arraystretch}{1} \setlength{\LTcapwidth}{4in} }

\newpage
\begin{quadroBase}[Código Tabela 3]
{\footnotesize
  \lstinputlisting[firstline=8]{../exemplo/tabelas/tabela-03-modelo.tex} 
}
\end{quadroBase}

% --- Importa a Tabela 4 ---
\clearpage
% Tabela Modelo 4
% Extraída ou adaptada de:
% IBGE, Fundação Instituto Brasileiro de Geografia e Estatística.
% Normas de apresentação tabular. 3. ed. Rio de Janeiro: IBGE, 1993.
%
% Este arquivo é destinado a ser incluído via \input{}

{\renewcommand{\arraystretch}{0.9}
\SingleSpacing
\ctable[
  label = tabIbge4,
  caption = {Esperança de vida ao nascer, por região socioeconômica -- Brasil -- 1940/1980},
  width = \linewidth, center, pos = !htb, notespar, nosuper
]
{X r r r r r } %  Definição das colunas
{% notas 
  \wIbgeFonte{Fundação Instituto Brasileiro de Geografia e Estatística (IBGE).}
  \wIbgeNota[Notas]{Média das esperanças de vida ao nascer, resultantes de interpolação linear, nas Tábuas de Mortalidade Modelo Brasil, das probabilidades de morrer até as idades de 2, 3 e 5 anos, obtidas através do emprego da Técnica de Brass. \par
    \wIbgeSinaisConvencionais{..,...}
  }
  \wIbgeNoteFont % Define a fonte para \tnote
  \tnote[(1)]{Estimativas sujeitas a revisão, por não estar concluído o processo de avaliação de consistência das informações sobre filhos tidos nascidos vivos e nascidos mortos, do Censo Demográfico de 1980.}
  \tnote[(2)]{Inclui a população das Regiões Norte e Centro-Oeste.}
  \tnote[(3)]{Exclui os dados da zona rural das Regiões Norte e Centro-Oeste.}
  \tnote[(4)]{Exclui os dados relativos à Região VII, uma vez que a Pesquisa Nacional por Amostra de Domicílios só foi estendida àquela região a partir de 1973.}

}
{ % conteúdo da tabela (Parte 1)
  \FL
  \multirow{2}{*}{Região socioeconômica} & \multicolumn{5}{c}{Esperança de vida ao nascer (anos)} \\ \cmidrule{2-6}
  & 1940 & 1950 & 1960 & 1970 & 1972  \ML

  \hspace{5mm} Brasil & \tmark[(2)] 42,74 &  \tmark[(2)]  45,90 & \tmark[(2)]  52,37 & \tmark[(3)] 52,49 & \tmark[(4)] 53,36 \\ \addlinespace 
  Região I - RJ & 45,38 & 50,91 & 59,19 & 57,29 & 63,21 \\
  Região II - SP & 43,57 & 49,92 & 59,11 & 58,45 & 64,35 \\
  Região III - PR, SC e RS & 50,09 & 53,33 & 60,34 & 60,26 & 63,77 \\
  Região IV - MG e ES & 43,93 & 47,10 & 53,29 & 54,78 & 60,38 \\
  Região V - MA, PI, CE, RN, PB, PE, AL, SE e BA & 38,17 & 38,69 & 43,51 & 44,38 & 42,55 \\ 
  Região VI - DF & .. & .. & 48,91 & 54,17 & 60,31 \\
  Região VII - RO, AC, AM, RR, PA, AP, MS, MT e GO & ... & ... & ... & 56,57 & ...  \LL
  
  
  % --- Repetição do Cabeçalho (Parte 2) ---
  \FL
  
  \multirow{2}{*}{Região socioeconômica} & \multicolumn{5}{c}{Esperança de vida ao nascer (anos)} \\ \cmidrule{2-6}
  & 1973 & 1976 & 1977 & 1978 & 1980 \tmark[(1)]  \ML

  \hspace{5mm}Brasil & \tmark[(3)] 54,56 & \tmark[(3)] 57,25 & \tmark[(3)] 57,81 & \tmark[(3)] 58,44 & \tmark[(4)] 59,83 \\ \addlinespace
  Região I - RJ & 63,44 & 65,96 & 65,08 & 64,81 & 63,23 \\
  Região II - SP &  64,87 & 64,31 & 64,54 & 64,98 & 63,55 \\ 
  Região III - PR, SC e RS & 60,28 & 63,57 & 63,90 & 64,05 & 66,98 \\
  Região IV - MG e ES & 60,27 & 61,82 & 61,12 & 63,50 & 62,20 \\
  Região V - MA, PI, CE, RN, PB, PE, AL, SE e BA & 42,76 & 47,51 & 48,93 & 48,94 & 51,57 \\
  Região VI - DF & 60,65 & 63,83 & 64,59 & 63,35 & 66,24 \\
  Região VII - RO, AC, AM, RR, PA, AP, MS, MT e GO & 65,93 & 62,44 & 61,85 & 62,53 & 64,30 \LL
}
\renewcommand{\arraystretch}{1.0}
}

\newpage
\begin{quadroBase}[Código Tabela 4]
{\footnotesize
  \lstinputlisting[firstline=8]{../exemplo/tabelas/tabela-04-modelo.tex} 
}
\end{quadroBase}

% --- Importa a Tabela 5 ---
\clearpage
% Tabela Modelo 5
% Extraída ou adaptada de:
% IBGE, Fundação Instituto Brasileiro de Geografia e Estatística.
% Normas de apresentação tabular. 3. ed. Rio de Janeiro: IBGE, 1993.
%
% Este arquivo é destinado a ser incluído via \input{}

\ctable[
  label = tabIbge5,
  caption = {Taxa de desemprego aberto (1), por Região Metropolitana, ano e mês de investigação -- Brasil -- janeiro de 1991 a maio de 1992}, 
  width = \linewidth, center, pos = !htb, notespar, nosuper
]
{X r r r r r r } %  Definição das colunas
{% notas
  \wIbgeFonte{Fundação Instituto Brasileiro de Geografia e Estatística -- Pesquisa Mensal de Emprego.}
  \wIbgeNoteFont % Define a fonte para \tnote 
  \tnote[(1)]{Percentual de pessoas de 15 anos ou mais de idade procurando trabalho, em relação às pessoas de 15 anos ou mais de idade economicamente ativas, na semana de referência.}
}
{ % conteúdo da tabela
  \FL
  \multirow{2}{*}{Ano e mês} & \multicolumn{6}{|c}{Região Metropolitana} \\ \cmidrule{2-7}
  & \multicolumn{1}{|c}{Recife} & \multicolumn{1}{|c}{Salvador} & \multicolumn{1}{|C{13mm}}{Belo Horizonte} & \multicolumn{1}{|C{10mm}}{Rio de Janeiro} & \multicolumn{1}{|C{10mm}}{São Paulo} & \multicolumn{1}{|C{10mm}}{Porto Alegre} \ML

  \multicolumn{7}{l}{\textbf{1991}} \\
  Janeiro \dotfill   & 86,10 & 5,43 & 4,77 & 4,24 & 5,91 & 4,56 \\
  Fevereiro \dotfill & 6,44  & 5,18 & 5,00 & 3,81 & 6,37 & 5,48 \\
  Março  \dotfill    & 6,33  & 5,76 & 5,06 & 4,24 & 7,22 & 5,14 \\
  Abril  \dotfill    & 6,67  & 6,06 & 4,47 & 4,13 & 6,93 & 5,44 \\
  Maio   \dotfill    & 6,21  & 7,26 & 4,61 & 4,54 & 6,49 & 5,04 \\
  Junho  \dotfill    & 5,30  & 6,43 & 4,31 & 3,63 & 5,61 & 3,90 \\
  Julho  \dotfill    & 4,46  & 6,52 & 3,18 & 2,55 & 4,34 & 3,15 \\
  Agosto  \dotfill   & 5,76  & 5,67 & 3,67 & 2,84 & 4,38 & 3,82 \\
  Setembro \dotfill  & 7,05  & 6,22 & 3,63 & 3,38 & 4,43 & 4,03 \\
  Outubro \dotfill   & 5,65  & 6,30 & 3,74 & 3,28 & 4,52 & 3,95 \\
  Novembro \dotfill  & 6,06  & 4,83 & 3,70 & 3,40 & 5,03 & 4,28 \\
  Dezembro  \dotfill & 4,72  & 5,23 & 3,15 & 3,04 & 4,98 & 3,33 \\
  \addlinespace
  \multicolumn{7}{l}{\textbf{1992}} \\
  Janeiro \dotfill   & 6,13  & 5,54 & 3,95 & 3,60 & 5,78 & 3,63 \\
  Fevereiro \dotfill & 8,35  & 6,38 & 5,76 & 4,43 & 7,58 & 5,43 \\
  Março   \dotfill   & 8,59  & 7,16 & 5,09 & 4,09 & 7,24 & 6,25 \\
  Abril  \dotfill    & 9,43  & 6,22 & 5,58 & 4,03 & 6,39 & 5,90 \\
  Maio   \dotfill    & 10,17 & 7,25 & 5,69 & 4,71 & 7,10 & 6,23 \LL
}

\newpage
\begin{quadroBase}[Código Tabela 5]
{\footnotesize
  \lstinputlisting[firstline=8]{../exemplo/tabelas/tabela-05-modelo.tex} 
}
\end{quadroBase}

% --- Importa a Tabela 6 ---
\clearpage
% Tabela Modelo 6
% Extraída ou adaptada de:
% IBGE, Fundação Instituto Brasileiro de Geografia e Estatística.
% Normas de apresentação tabular. 3. ed. Rio de Janeiro: IBGE, 1993.
%
% Este arquivo é destinado a ser incluído via \input{}

\begin{SingleSpace}
\ctable[
  label = tabIbge6,
  caption = {Turistas estrangeiros, por meio de transporte e a Unidade da Federação de entrada -- Brasil -- 1989-1990}, 
  width = \linewidth, center, pos = !htb, notespar, nosuper
]
{L{2mm} L{25mm} R{10mm} R{10mm} || L{2mm} L{25mm} R{10mm} R{10mm} }
{% notas
  \wIbgeFonte[Fontes]{Presidência da República, Secretaria de Desenvolvimento Regional, Instituto Brasileiro de Turismo, Divisão de Estatística -- Departamento de Polícia Federal.}
}
{ % conteúdo da tabela 
  \FL
  \multicolumn{2}{C{27mm}|}{\multirow{2}{*}{\parbox{30mm}{\centering Meio de Transporte e Unidade da Federação de entrada}}}	&
  \multicolumn{2}{C{20mm}||}{Entrada de turista estrangeiro} &
  \multicolumn{2}{C{27mm}|}{\multirow{2}{*}{\parbox{30mm}{\centering Meio de Transporte e Unidade da Federação de entrada}}}	&
  \multicolumn{2}{C{20mm}}{Entrada de turista estrangeiro} 
  \\ \cmidrule{3-4} \cmidrule{7-8} 
  & & \multicolumn{1}{|r|}{1989} & 1990 & &  & \multicolumn{1}{|r|}{1989} & 1990 \ML

  & \hspace{5mm}Total & 1402897 & 1078601 & \multicolumn{2}{l}{Via maritima} & 24612 & 39070 \\
  & Amazonas & 13032 & 11789 & & Pernambuco & 3513 & 5043 \\
  & Pará & 16882 & 18669 & & Bahia & 2245 & 2828 \\
  & Pernambuco & 34541 & 38935 & & Rio de Janeiro & 5416 & 12178 \\
  & Bahia & 16882 & 16208 & & São Paulo & 590 & 852 \\
  & Rio de Janeiro & 472445 & 438015 & & Paraná & 1203 & 4920 \\
  & São Paulo & 183960 & 150810 & & Rio Grande do Sul & 2062 & 3234 \\
  & Paraná & 159779 & 122830 & & Outras & 9583 & 10015 \\

  & Rio Grande do Sul & 422658 & 225247 & & & & \\
  & Mato Grosso do Sul & 32742 & 23550 & \multicolumn{2}{l}{Via terrestre} & 592933 & 358743 \\
  & Distrito Federal & 414 & 461 & & Amazonas & 2023 & 1611 \\
  & Outras & 49562 & 32087 & & Paraná & 147252 & 108024 \\
  & & & & & Rio Grande do Sul & 380204 & 207893 \\
  \multicolumn{2}{l}{Via aérea} & 748021 & 665695 & &Mato Grosso do Sul & 32623 & 23454 \\
  & Amazonas & 9800 & 9476 & &  Outras & 30831 & 17761\\
  & Pará & 14617 & 15626 & &  \\
  & Pernambuco & 31028 & 33892 & & & & \\
  & Bahia & 14637 & 13380 & \multicolumn{2}{l}{Via fluvial} & 37331 & 15093   \\
  & Rio de Janeiro & 467029 & 425837 & &  \\
  & São Paulo & 183370 & 149958 & \\
  & Paraná & 10011 & 9131 & & Amazonas & 1209 & 702 \\
  & Rio Grande do Sul & 8108 & 3528 & &  Pará & 2265 & 3043\\
  & Mato Grosso do Sul & 119 & 96 & &  Paraná & 1313 & 755\\
  & Distrito Federal & 414 & 461 & & Rio Grande do Sul & 32284 & 10592 \\
  & Outras & 8888 & 4310 & & Outras & 260 & 1 \\ \FL
}
\end{SingleSpace}


\begin{quadroBase}[Código Tabela 6]
{\footnotesize
  \lstinputlisting[firstline=8]{../exemplo/tabelas/tabela-06-modelo.tex} 
}
\end{quadroBase}

% --- Importa a Tabela 7 ---
\clearpage
% Tabela Modelo 7
% Extraída ou adaptada de:
% IBGE, Fundação Instituto Brasileiro de Geografia e Estatística.
% Normas de apresentação tabular. 3. ed. Rio de Janeiro: IBGE, 1993.
%
% Este arquivo é destinado a ser incluído via \input{}

{\renewcommand{\arraystretch}{0.9} 
\SingleSpacing
\ctable[
  label = tabIbge7,
  caption = {Preço médio de produtos e serviços selecionados -- INPC, Região Metropolitana de Belém (JUN/DEZ 1989--JUNHO/DEZ 1990)},
  width = \linewidth, center, pos = !htb, notespar, nosuper
]
{X C{15mm} r r r r} %  Definição das colunas
{% notas
   \wIbgeFonte{IBGE, Diretoria de Pesquisas, Departamento de Índices de Preços, Sistema Nacional de Índices de Preços ao consumidor.}
   \wIbgeNota{A partir de março de 1990 o padrão monetário mudou de cruzado novo (NCz\$) para cruzeiro (Cr\$).}
}
{ % conteúdo da tabela
   \FL
   \multicolumn{1}{R{30mm}|}{\multirow{3}{*}{\parbox{30mm}{Produto e serviço selecionado}}} & \multicolumn{1}{R{15mm}}{\multirow{3}{*}{\parbox{15mm}{Unidade de medida}}} & \multicolumn{4}{|r}{Preço médio} \\ \cmidrule{3-6}
   \multicolumn{1}{l|}{} && \multicolumn{2}{|r}{1989 (NCz\$)} & \multicolumn{2}{|r}{1990 (NCz\$)} \\ \cmidrule{3-6}
   \multicolumn{1}{l|}{}&& \multicolumn{1}{|r}{Junho} & \multicolumn{1}{|r}{Dezembro} & \multicolumn{1}{|r}{Junho} & \multicolumn{1}{|r}{Dezembro} \ML
   \multicolumn{6}{c}{\textbf{Alimentícios}}\\
   Açúcar refinado  &  kg  &  0,61  &  7,04  &  31,92  &  74,81 \\ 
   Alface  &  unidade  &  1,16  &  4,20  &  43,12  &  80,69 \\ 
   Arroz  &  5 kg  &  0,82  &  5,32  &  38,19  &  134,96 \\ 
   Banana-prata  &  dúzia  &  1,22  &  4,93  &  58,05  &  117,57 \\ 
   Batata-inglesa  &  kg  &  1,75  &  3,94  &  44,83  &  113,11 \\ 
   Café moído  &  250 g  &  1,61  &  8,73  &  68,75  &  99,12 \\ 
   Carne de porco com osso  &  kg  &  5,01  &  29,06  &  205,00  &  421,66 \\ 
   Carne-seca  &  kg  &  5,82  &  24,48  &  201,38  &  363,46 \\ 
   Cebola  &  kg  &  0,85  &  7,47  &  129,36  &  62,79 \\ 
   Cerveja  &  600 ml  &  1,02  &  9,52  &  58,23  &  167,36 \\ 
   Chá-de-dentro  &  kg  &  6,53  &  29,10  &  237,80  &  420,44 \\ 
   Farinha de mandioca  &  L  &  0,37  &  2,08  &  16,75  &  61,59 \\ 
   Feijão (tipo mais comercializado)  &  kg  &  2,10  &  8,61  &  69,60  &  118,49 \\ 
   Fígado  &  kg  &  5,68  &  22,66  &  166,87  &  359,34 \\ 
   Frango  &  kg  &  3,44  &  17,09  &  90,30  &  215,79 \\ 
   Leite em pó integral  &  454 g  &  2,11  &  19,95  &  137,07  &  318,81 \\ 
   Macarrão sem ovos  &  500 g  &  0,65  &  6,03  &  36,56  &  71,11 \\ 
   Óleo de soja  &  900 ml  &  1,20  &  6,70  &  49,39  &  117,22 \\ 
   Ovos  &  dúzia  &  2,41  &  9,35  &  62,52  &  116,60 \\ 
   Pá com osso  &  kg  &  4,30  &  18,47  &  139,68  &  262,01 \\ 
   Pão francês  &  200 g  &  0,24  &  2,12  &  13,15  &  27,30 \\ 
   Peixe corvina  &  kg  &  3,14  &  14,00  &  140,71  &  302,75 \\ 
   Tomate  &  kg  &  1,23  &  5,57  &  80,52  &  104,51 \\ 
   \multicolumn{6}{c}{\textbf{Não alimentícios}}\\
   Álcool combustível  &  L  &  0,46  &  3,84  &  28,60  &  59,07 \\ 
   Botijão de gás  &  13 kg  &  2,73  &  29,18  &  230,93  &  510,12 \\ 
   Cigarro  &  maço  &  0,73  &  4,89  &  43,83  &  87,00 \\ 
   Energia elétrica (consumo médio)  &    &  3,09  &  48,42  &  361,94  &  691,73 \\ 
   Gasolina  &  L  &  0,62  &  5,11  &  38,00  &  78,65 \\ 
   Ônibus urbano  &   &  0,17  &  1,34  &  9,12  &  27,50 \\ 
   Taxa de água e esgoto (consumo médio)  &    &  10,80  &  93,80  &  243,76  &  1\,059,82 \\ 
   Táxi (corrida padrão)  &    &  2,52  &  24,75  &  144,70  &  420,20 \LL 
}
}



\begin{quadroBase}[Código Tabela 7]
{\footnotesize
  \lstinputlisting[firstline=8]{../exemplo/tabelas/tabela-07-modelo.tex} 
}
\end{quadroBase}

% --- Importa a Tabela 8 ---
\clearpage
% Tabela Modelo 8
% Extraída ou adaptada de:
% IBGE, Fundação Instituto Brasileiro de Geografia e Estatística.
% Normas de apresentação tabular. 3. ed. Rio de Janeiro: IBGE, 1993.
%
% Este arquivo é destinado a ser incluído via \input{}

\ctable[
  label = tabIbge8,
  caption = {Superfície total, em números absolutos e relativos, por zona hipsométrica do Brasil -- 1973}, 
  width = \linewidth, center, pos = th, notespar, nosuper
]
{X R{30mm} R{30mm}} %  Definição das colunas
{% notas
  \wIbgeFonte{IBGE, Diretoria de Geociências, Departamento de Cartografia.}
  \wIbgeNota{Dados sujeitos a retificação.}
  \wIbgeNoteFont % Define a fonte para \tnote 
  \tnote[(1)]{ Áreas de reservas ecológicas, conforme resolução n. 04 de 18.09.1985 do Conselho Nacional do Meio Ambiente,} 
}
{ % conteúdo da tabela
  \FL
  \multirow{2}{*}{Zona hipsométrica (m)} & \multicolumn{2}{c}{Superfície total} \\
  &  Absoluta (km\textsuperscript{2}) & Relativa (\%)  \ML

  \hspace{5mm} Total & 8 511 996 & 10 000 \\ \addlinespace
  
  Terras baixas & 3 489 553 & 4 100 \\
  0 a 100 & 2 050 318 & 2 409 \\
  101 a 200 & 1 439 235 & 1 691 \\ \addlinespace
  
  Terras altas & 4 976 176 & 5 846 \\
  201 a 500 & 3 151 646 & 3 703 \\
  501 a 800 & 1 249 906 & 1 468 \\
  801 a 1200 & 574 624 & 675 \\ \addlinespace
  
  Áreas culminantes & 46 267 & 0,54 \\
  1201 a 1800 & 44 767 & 0,52 \\
  1801 a 3014\tmark[(1)] & 1500 & 0,02 \LL
}

\newpage
\begin{quadroBase}[Código Tabela 8]
{\footnotesize
  \lstinputlisting[firstline=8]{../exemplo/tabelas/tabela-08-modelo.tex} 
}
\end{quadroBase}

% --- Importa a Tabela 9 ---
\clearpage
% Tabela Modelo 9
% Extraída ou adaptada de:
% IBGE, Fundação Instituto Brasileiro de Geografia e Estatística.
% Normas de apresentação tabular. 3. ed. Rio de Janeiro: IBGE, 1993.
%
% Este arquivo é destinado a ser incluído via \input{}

\ctable[
  caption = {Número de estabelecimentos agropecuários, pessoal ocupado, número de tratores e efetivo de bovinos, por grupo de densidade do rebanho bovino -- Brasil -- 1975}, 
  label = tabIbge9,
  width = \linewidth, center, pos = h, notespar, nosuper
]
{X R{15mm} R{15mm} R{15mm} R{15mm}} %  Definição das colunas
{% notas
  \wIbgeFonte{IBGE, Diretoria de Pesquisas, Coordenação dos Censos Econômicos, Censo Agropecuário.}
  \wIbgeNota{Dados sujeitos a retificação.}
}
{ % conteúdo da tabela
  \FL
  Grupos de densidade do rebanho bovino &
  \multicolumn{1}{|C{15mm}|}{N. de estabelecimentos} &
  Pessoal ocupado &
  \multicolumn{1}{|C{15mm}|}{N. de tratores} &
  Efetivo de bovinos \ML
  
  Total & 5\,834\,779 & 23\,273\,517 & 652\,049 & 127\,643\,292 \\
  Menos de 15 bovinos por km\textsuperscript{2} & 1\,989\,702 & 7\,817\,021 & 71\,288 & 20\,680\,255 \\
  15 a menos de 30 bovinos por km\textsuperscript{2} & 1\,298\,248 & 5\,549\,210 & 125\,569 & 25\,039\,093 \\
  30 a menos de 50 bovinos por km\textsuperscript{2} & 1\,741\,958 & 6\,677\,749 & 258\,611 & 39\,228\,726 \\
  50 e mais bovinos por kmv & 804\,871 & 3\,229\,537 & 196\,581 & 42\,695\,218 \LL
}


\begin{quadroBase}[Código Tabela 9]
{\footnotesize
  \lstinputlisting[firstline=8]{../exemplo/tabelas/tabela-09-modelo.tex} 
}
\end{quadroBase}

% --- Importa a Tabela 10 ---
\clearpage
% Tabela Modelo 10
% Extraída ou adaptada de:
% IBGE, Fundação Instituto Brasileiro de Geografia e Estatística.
% Normas de apresentação tabular. 3. ed. Rio de Janeiro: IBGE, 1993.
%
% Este arquivo é destinado a ser incluído via \input{}

{\SingleSpacing
\ctable[
  label = tabIbge10,
  caption = {População de O a 11 meses de idade, por aleitamento materno, grupo de idade e classe de rendimento mensal familiar per capita Brasil -- 1989}, 
  width = \linewidth, center, pos = !htb, notespar, nosuper
]
{L{28mm} R{12mm} R{12mm} R{12mm} R{12mm} R{12mm}  R{12mm}} %  Definição das colunas
{% notas  
  \wIbgeFonte[Fontes]{IBGE, Diretoria de Pesquisas, Departamento de Estatísticas e Indicadores Sociais - Instituto Nacional de Alimentação e Nutrição, Pesquisa Nacional de Saúde e Nutrição.}
  \wIbgeNota{1 Exclui as pessoas cuja condição na família era pensionista, empregado doméstico ou parente do empregado doméstico. \par
  2 Exclui a população da zona rural de RO, AC, AM, RR, PA AP \par
  3 \wIbgeSinaisConvencionais{-}
  }
  \wIbgeNoteFont % Define a fonte para \tnote
  \tnote[(1)]{ Inclui sem declaração de aleitamento materno.} 
}
{ % conteúdo da tabela (Parte 1)
  \FL
  \multicolumn{1}{C{28mm}}{\multirow{4}{*}{\parbox{28mm}{Classe de rendimento mensal familiar per capita (salário mínimo)}}} & \multicolumn{6}{|C{66mm}}{População 0 a 11 meses} \\ \cmidrule{2-7}

& \multicolumn{1}{|c}{\multirow{3}{*}{Total \tmark[(1)]}}  & \multicolumn{5}{|c}{Amamentada} \\ \cmidrule{3-7}
  & \multicolumn{1}{|c}{} & \multicolumn{1}{|c}{\multirow{2}{*}{Total}}  & \multicolumn{4}{|c}{Grupo de idade } \\ \cmidrule{4-7}
   & \multicolumn{1}{|c}{}& \multicolumn{1}{|c}{}& \multicolumn{1}{|C{12mm}}{Menos de 1 mês} & 1 a 4 meses & 5 a 8 meses & 9 a 11 meses \ML

   
Até 1/4\dotfill & 406012 & 261275 & 23848 & 95133 & 65332 & 86962 \\
Mais de 1/4 a 1/2 \dotfill& 615162 & 358192 & 42964 & 138560 & 101122 & 75546 \\
Mais de 1/2 a 1 \dotfill& 727327 & 397544 & 51269 & 174828 & 86585 & 85062 \\
Mais de 1 a 2 \dotfill& 622383 & 291112 & 37710 & 127929 & 49185 & 76288 \\
Mais de 2 \dotfill& 560765 & 256331 & 31494 & 128274 & 67031 & 29532 \\
Sem rendimento \dotfill& 266590 & 139138 & 36992 & 55752 & 35748 & 10646 \LL

& \\

% --- Repetição do Cabeçalho (Parte 2) ---
\FL
  \multicolumn{1}{C{28mm}}{\multirow{4}{*}{\parbox{28mm}{Classe de rendimento mensal familiar per capita (salário mínimo)}}} & \multicolumn{6}{|C{66mm}}{População 0 a 11 meses} \\ \cmidrule{2-6}

&  \multicolumn{6}{|c}{Não-amamentada} \\ \cmidrule{2-6}
   & \multicolumn{1}{|c}{\multirow{2}{*}{Total}}  & \multicolumn{4}{|c}{Grupo de idade } \\ \cmidrule{3-6}
   & \multicolumn{1}{|c}{}& \multicolumn{1}{|C{12mm}}{Menos de 1 mês} & 1 a 4 meses & 5 a 8 meses & 9 a 11 meses \\ \cmidrule{1-6}

Até 1/4 \dotfill & 144737 & 3889 & 28112 & 58297 & 54439 \\
Mais de 1/4 a 1/2 \dotfill & 256970 & 2322 & 58162 & 110161 & 86325 \\
Mais de 1/2 a 1 \dotfill & 329783 & 11394 & 82837 & 148709 & 86843 \\
Mais de 1 a 2 \dotfill & 327463 & 13482 & 73666 & 125599 & 114716 \\
Mais de 2 \dotfill & 300899 & - & 57296 & 106821 & 136782 \\
Sem rendimento \dotfill & 127452 & 3820 & 31885 & 60075 & 31672 \LL
  
}
}


\begin{quadroBase}[Código Tabela 10]
{\footnotesize
  \lstinputlisting[firstline=8]{../exemplo/tabelas/tabela-10-modelo.tex} 
}
\end{quadroBase}

% --- Importa a Tabela 11 ---
\clearpage
% Tabela Modelo 11
% Extraída ou adaptada de:
% IBGE, Fundação Instituto Brasileiro de Geografia e Estatística.
% Normas de apresentação tabular. 3. ed. Rio de Janeiro: IBGE, 1993.
%
% Este arquivo é destinado a ser incluído via \input{}

{
\SingleSpacing
\ctable[
  label = tabIbge11,
  caption = {Número de registros no ano de nascidos vivos, com indicação dos nascidos no ano, por sexo e Grande Região de registro -- Brasil -- 1987--1989}, 
  width = \linewidth, center, pos = !htb, notespar, nosuper
]
{X c r r r r} %  Definição das colunas
{% notas 
  \wIbgeFonte{IBGE, Diretoria de Pesquisas, Departamento de População, pesquisa do Registro Civil.}
  \wIbgeNoteFont % Define a fonte para \tnote
  \tnote[(1)]{Inclui registros de nascidos vivos em anos anteriores.} 
}
{ % conteúdo da tabela
  \FL
  \multicolumn{1}{l}{\multirow{4}{*}{Grande Região de registro}} & \multicolumn{1}{|c|}{\multirow{4}{*}{Ano}} & \multicolumn{4}{c}{Nascidos vivos registrados no ano} \\ \cmidrule(lr){3-6}
  & \multicolumn{1}{|c|}{} & \multirow{3}{*}{Total\tmark[(1)]} & \multicolumn{3}{|c}{Nascidos no ano} \\ \cmidrule(lr){4-6}
  & \multicolumn{1}{|c|}{}& & \multicolumn{1}{|c}{\multirow{2}{*}{Total}} & 	\multicolumn{2}{|c}{Sexo} \\ \cmidrule(lr){5-6}
  & \multicolumn{1}{|c|}{}& &  \multicolumn{1}{|c}{}& \multicolumn{1}{|c|}{Masculino} & Feminino \ML

 \multirow{3}{*}{\hspace{5mm} Brasil} & 1987 & 4\,072\,032 & 2\,660\,886 & 1\,358\,475 & 1\,302\,411 \\
 & 1988 & 4\,993\,923 & 2\,809\,657 & 1\,432\,295 & 1\,377\,362 \\
 & 1989 & 3\,636\,901 & 2\,581\,035 & 1\,317\,159 & 1\,263\,876 \\
\multicolumn{6}{l}{} \\

\multirow{3}{*}{Norte} & 1987 & 288\,496 & 104\,706 & 53\,389 & 51\,317 \\
 & 1988 & 486\,678 & 121\,683 & 62\,006 & 59\,677 \\
 & 1989 & 302\,123 & 112\,965 & 57\,252 & 55\,713 \\
\multicolumn{6}{l}{} \\

\multirow{3}{*}{Nordeste} & 1987 & 1\,500\,769 & 681\,288 & 346\,248 & 335\,040 \\
 & 1988 & 2\,023\,058 & 738\,017 & 374\,660 & 363\,357 \\
 & 1989 & 1\,132\,531 & 599\,608 & 305\,636 & 293\,972 \\
\multicolumn{6}{l}{} \\

\multirow{3}{*}{Sudeste} & 1987 & 1\,483\,761 & 1\,252\,226 & 640\,542 & 611\,684 \\
 & 1988 & 1\,567\,884 & 1\,293\,873 & 660\,200 & 633\,673 \\
 & 1989 & 1\,455\,218 & 1\,253\,513 & 640\,008 & 613\,505 \\

 \multicolumn{6}{l}{} \\
\multirow{3}{*}{Sul} & 1987 & 527\,081 & 445\,655 & 227\,804 & 217\,851 \\
 & 1988 & 568\,199 & 467\,856 & 239\,470 & 228\,386 \\
 & 1989 & 509\,311 & 446\,285 & 227\,972 & 218\,313 \\

 \multicolumn{6}{l}{} \\
\multirow{3}{*}{Centro-Oeste} & 1987 & 271\,925 & 177\,011 & 90\,492 & 86\,519 \\
 & 1988 & 348\,104 & 188\,228 & 95\,959 & 92\,269 \\
 & 1989 & 237\,718 & 168\,664 & 86\,291 & 82\,373 \LL
}
}


\begin{quadroBase}[Código Tabela 11]
{\footnotesize
  \lstinputlisting[firstline=8]{../exemplo/tabelas/tabela-11-modelo.tex} 
}
\end{quadroBase}

% --- Importa a Tabela 12 ---
\clearpage
% Tabela Modelo 12
% Extraída ou adaptada de:
% IBGE, Fundação Instituto Brasileiro de Geografia e Estatística.
% Normas de apresentação tabular. 3. ed. Rio de Janeiro: IBGE, 1993.
%
% Este arquivo é destinado a ser incluído via \input{}

\ctable[
  label = tabIbge12,
  caption = {População de 5 anos ou mais de idade, por sexo e grupo de idade -- Brasil -- 1990}, 
  width = \linewidth, center, pos = !htb, notespar, nosuper
]
{X r r r r r}%  Definição das colunas
{% notas 
  \wIbgeFonte{IBGE, Diretoria de Pesquisas, Departamento de Emprego e Rendimento.}
  \wIbgeNota{As diferenças entre soma de parcelas e respectivos totais são provenientes do critério de arredondamento. \par
  Exclui as pessoas da zona rural da Região Norte, sem Tocantins. \par
    \wIbgeSinaisConvencionais{0, -}
  }
  \wIbgeNoteFont % Define a fonte para \tnote
  \tnote[(1)]{Inclui sem declaração de alfabetização.}
}
{ % conteúdo da tabela
  \FL
\multirow{4}{*}{Grupo de idade} & \multicolumn{5}{|c}{População de 5 anos ou mais de idade (1\,000)} \\ \cmidrule{2-6}
& \multicolumn{1}{|c|}{Total \tmark[(1)]} & \multicolumn{4}{c}{ Condição de alfabetização}  \\ \cmidrule{3-6}
& \multicolumn{1}{|c|}{} & \multicolumn{2}{c|}{Alfabetizada} & \multicolumn{2}{c}{Não-alfabetizada} \\ \cmidrule{3-6}
& \multicolumn{1}{|c|}{} & Homem & \multicolumn{1}{|c|}{Mulher} & Homem & \multicolumn{1}{|c}{Mulher } \ML

\hspace{5mm}Total \dotfill        & 131\,317 & 48\,926 & 51\,796 & 15\,318 & 15\,276 \\ \addlinespace
5 a 6 anos \dotfill        & 6\,772   & 287     & 313     & 3\,202  & 2\,970  \\ \addlinespace
7 a 9 anos  \dotfill       & 10\,916  & 3\,240  & 3\,430  & 2\,258  & 1\,985  \\ \addlinespace
10 a 14 anos  \dotfill     & 16\,981  & 7\,029  & 7\,507  & 1\,489  & 957     \\ \addlinespace
15 a 19 anos \dotfill      & 14\,915  & 6\,580  & 6\,929  & 929     & 476     \\ \addlinespace
20 a 24 anos \dotfill      & 13\,051  & 5\,707  & 6\,067  & 734     & 543     \\ \addlinespace
25 a 29 anos  \dotfill     & 12\,082  & 5\,077  & 5\,777  & 674     & 553     \\ \addlinespace
30 a 39 anos \dotfill      & 20\,679  & 8\,655  & 9\,272  & 1\,303  & 1\,448  \\ \addlinespace
40 a 49 anos \dotfill      & 14\,449  & 5\,556  & 5\,714  & 1\,435  & 1\,744  \\ \addlinespace
50 a 59 anos  \dotfill     & 10\,145  & 3\,664  & 3\,553  & 1\,245  & 1\,683  \\ \addlinespace
60 anos ou mais  \dotfill  & 11\,327  & 3\,129  & 3\,234  & 2\,049  & 2\,915  \\ \addlinespace
Idade ignorada   \dotfill  & 1        & 0       & -       & -       & 1       \LL
}

\newpage
\begin{quadroBase}[Código Tabela 12]
{\footnotesize
  \lstinputlisting[firstline=8]{../exemplo/tabelas/tabela-12-modelo.tex} 
}
\end{quadroBase}

% --- Importa a Tabela 13 ---
\clearpage
% Tabela Modelo 13
% Extraída ou adaptada de:
% IBGE, Fundação Instituto Brasileiro de Geografia e Estatística.
% Normas de apresentação tabular. 3. ed. Rio de Janeiro: IBGE, 1993.
%
% Este arquivo é destinado a ser incluído via \input{}

{\renewcommand{\arraystretch}{0.9}%
 \renewcommand{\cellalign}{cc}%
 \renewcommand{\cellgape}{}% evita espaço extra do makecell no cabeçalho
 \SingleSpacing
 \ctable[
   label=tabIbge13,
   caption={População residente em 1980 e 1991, por sexo, área total e densidade demográfica em 1991, para os Municípios do Estado de Roraima},
   width=\linewidth, center, pos=!htb, notespar, nosuper
 ]{X R{15mm} R{15mm} R{20mm} R{20mm}}{ 
   \wIbgeFonte{Sinopse Preliminar do Censo Demográfico 1991. Roraima, Amapá. Rio de Janeiro: IBGE n. 4, 31p.}
   \wIbgeNota{
     \wIbgeSinaisConvencionais{.., ...}
   }
   \wIbgeNoteFont % Define a fonte para \tnote
   \tnote[(1)]{Valores numéricos de áreas sujeitos a verificação/alteração, face ao processo de implantação de nova metodologia na medição.}
   \tnote[(2)]{Município instalado entre 01.09.1980 e 01.09.1991.}
   \tnote[(3)]{Município que sofreu desmembramento entre 01.09.1980 e 01.09.1991.}
 }{
   \FL
    \multirow{4}{=}{Município e sexo}    & 
        \multicolumn{2}{C{30mm}}{\multirow{2}{*}{População residente}}
      & \multirow{4}{=}{\parbox{20mm}{Área total em  01.09.1991 (km\textsuperscript{2}) \tmark[(1)]}}
      & \multirow{4}{=}{\parbox{20mm}{Densidade  demográfica em 01.09.1991 (hab/km\textsuperscript{2})}}     \\
     &&&& \\ 
     & 01.09.1980 & 01.09.1991 & & \\
     
 &&&&  \\ 
     \midrule
   
   Total \dotfill & 79\,159 & 215\,950 & 224\,131,3 & 0.96 \\
\multicolumn{5}{l}{}\\ 
Homem \dotfill & ... & 119\,170 & .. & .. \\ 
Mulher \dotfill & ... & 96\,780 & .. & .. \\ 
Alto Alegre \tmark[(2)] \dotfill & 3\,475 & 11\,196 & 25\,653,3 & 0,44 \\ 
Homem \dotfill & ... & 6\,889 & .. & .. \\ 
Mulher \dotfill & ... & 4\,307 & .. & .. \\
Boa Vista \tmark[(3)] \dotfill & 51\,662 & 142\,902 & 44\,295,0 & 3,23 \\
Homem \dotfill & ... & 76\,949 & .. & .. \\
Mulher \dotfill & ... & 65\,953 & .. & .. \\ 
Bonfim \tmark[(2)] \dotfill & 4\,524 & 9\,454 & 14\,390,0 & 0,66 \\ 
Homem \dotfill & ... & 5\,126 & .. & .. \\
Mulher \dotfill & ... & 4\,328 & .. & .. \\ 
Caracaraí \tmark[(3)] \dotfill & 4\,990 & 8\,910 & 51\,99,3 & 0,17 \\ 
Homem \dotfill & ... & 4\,859 & .. & .. \\ 
Mulher \dotfill & ... & 4\,051 & .. & .. \\ 
Mucajaí \dotfill & 3\,163 & 13\,135 & 23\,6017 & 56 \\
Homem \dotfill & ... & 8\,105 & .. & .. \\ 
Mulher \dotfill & ... & 5\,030 & .. & .. \\ 
Normandia \tmark[(2)] \dotfill & 7\,713 & 11\,165 & 12\,927,0 & 0,86 \\ 
Homem \dotfill & ... & 6\,291 & .. & .. \\ 
Mulher \dotfill & ... & 4\,874 & .. & .. \\
São João da Baliza \tmark[(2)] \dotfill & 1\,531 & 10\,089 & 19\,272,5 & 0,52 \\ 
Homem \dotfill & ... & 6\,104 & .. & .. \\ 
Mulher \dotfill & ... & 3\,985 & .. & .. \\ 
São Luiz \tmark[(2)] \dotfill & 2\,101 & 9\,099 & 32\,192,5 & 0,28 \\
Homem \dotfill & ... & 4\,847 & .. & .. \\ 
Mulher \dotfill & ... & 4\,252 & .. & .. \LL
   
 }
 \renewcommand{\arraystretch}{1.0}%
}


\begin{quadroBase}[Código Tabela 13]
{\footnotesize
  \lstinputlisting[firstline=8]{../exemplo/tabelas/tabela-13-modelo.tex} 
}
\end{quadroBase}

% --- Importa a Tabela 14 ---
\clearpage
% Tabela Modelo 14
% Extraída ou adaptada de:
% IBGE, Fundação Instituto Brasileiro de Geografia e Estatística.
% Normas de apresentação tabular. 3. ed. Rio de Janeiro: IBGE, 1993.
%
% Este arquivo é destinado a ser incluído via \input{}

{\renewcommand{\arraystretch}{0.9} 
\SingleSpacing
\ctable[
  label = tabIbge14,
  caption = {Total de estabelecimentos, pessoal ocupado, valor da produção e valor da transformação industrial das indústrias metalúrgicas, por Unidade da Federação do Brasil -- 1982}, 
  width = \linewidth, center, pos = !htb, notespar, nosuper
]
{L{35mm} Y Y Y Y Y } %  Definição das colunas
{% notas 
    \wIbgeFonte{Pesquisa Industrial - 1982--1984. Dados gerais, Brasil. Río de Janeiro: IBGE, v. 9, 410p.}
    \wIbgeNota{
      \wIbgeSinaisConvencionais{x, -}
    }
   \wIbgeNoteFont % Define a fonte para \tnote
   \tnote[(1)]{Em 31.12.1982.}
   \tnote[(2)]{Inclui o valor dos serviços prestados a terceiros e a estabelecimentos da mesma empresa.}
}
{ % conteúdo da tabela
  \FL
    Unidade da Federação & Total de estabelecimentos &     Pessoal ocupado \tmark[(1)] & Valor da produção (1\,000 Cr\$) \tmark[(2)] & Valor da transformação industrial. (1\,000 Cr\$)
    \ML

Brasil \dotfill & 8\,452 & 448\,932 & 4\,637\,512 & 646\,043 \\
\multicolumn{5}{l}{} \\
Rondônia \dotfill & 1 & x & x & x \\
Acre \dotfill & 2 & x & x & x \\
Amazonas \dotfill & 31 & 1\,710 & 21\,585 & 10\,103 \\
Roraima \dotfill & 2 & x & x & x \\
Pará \dotfill & 43 & 1\,675 & 6\,492 & 3\,287 \\
Amapá \dotfill & - & - & - & - \\
\multicolumn{5}{l}{} \\
Maranhão \dotfill & 14 & 328 & 498 & 251 \\
Piauí \dotfill & 12 & 193 & 454 & 159 \\
Ceará \dotfill & 74 & 5\,336 & 21\,732 & 10\,878 \\
Rio Grande do Norte \dotfill & 11 & 343 & 1\,267 & 383 \\
Paraíba \dotfill & 30 & 794 & 2\,089 & 1\,265 \\
Pernambuco \dotfill & 105 & 5\,171 & 44\,873 & 14\,506 \\
Alagoas \dotfill & 20 & 439 & 4\,101 & 1\,768 \\
Sergipe \dotfill & 20 & 423 & 1\,447 & 534 \\
Bahia \dotfill & 116 & 5\,527 & 89\,072 & 27\,679 \\
\multicolumn{5}{l}{} \\
Minas Gerais \dotfill & 736 & 54\,264 & 954\,258 & 306\,856 \\
Espírito Santo \dotfill & 42 & 2\,281 & 22\,923 & 6\,297 \\
Rio de Janeiro \dotfill & 847 & 40\,768 & 635\,731 & 177\,358 \\
São Paulo \dotfill & 4\,699 & 272\,983 & 2\,531\,363 & 939\,032 \\
\multicolumn{5}{l}{} \\
Paraná \dotfill & 449 & 11\,188 & 43\,797 & 22\,014 \\
Santa Catarina \dotfill & 305 & 10\,816 & 84\,294 & 41\,894 \\
Rio Grande do Sul \dotfill & 706 & 30\,103 & 156\,680 & 74\,316 \\
\multicolumn{5}{l}{} \\
Mato Grosso do Sul \dotfill & 29 & 485 & 1\,643 & 623 \\
Mato Grosso \dotfill & 13 & 528 & 884 & 686 \\
Goiás \dotfill & 106 & 2\,686 & 9\,860 & 4\,800 \\
Distrito Federal \dotfill & 28 & 843 & 2\,577 & 1\,301 \LL

}
\renewcommand{\arraystretch}{1.0} 
}


\begin{quadroBase}[Código Tabela 14]
{\footnotesize
  \lstinputlisting[firstline=8]{../exemplo/tabelas/tabela-14-modelo.tex} 
}
\end{quadroBase}

% --- Importa a Tabela 15 ---
\clearpage
% Tabela Modelo 15
% Extraída ou adaptada de:
% IBGE, Fundação Instituto Brasileiro de Geografia e Estatística.
% Normas de apresentação tabular. 3. ed. Rio de Janeiro: IBGE, 1993.
%
% Este arquivo é destinado a ser incluído via \input{}

\ctable[
  label = tabIbge15,
  caption = {Altitude e coordenadas geográficas dos pontos mais altos do Brasil -- 1992}, 
  width = \linewidth, center, pos = !htb, notespar, nosuper
]
{L{30mm} X R{15mm} r r} %  Definição das colunas
{% notas  
  \wIbgeFonte{IBGE, Diretoria de Geociências, Departamento de Cartografia.}
  \wIbgeNota{Foram considerados os pontos com altura superior a 2 500 metros.}
  \wIbgeNoteFont % Define a fonte para \tnote
  \tnote[(1)]{As altitudes ao decímetro correspondem às medições de campo e, as demais, à leitura de cartas fopográficas.}
  \tnote[(2)]{Fronteira com a Venezuela.}
  \tnote[(3)]{ Fronteira com a Guiana.}
}
{ % conteúdo da tabela
  \FL
      \multirow{2}{*}{Topônimo} & \multirow{2}{*}{Localização} & 
        \multicolumn{1}{|c|}{\multirow{2}{*}{\parbox{15mm}{Altitude (m) \tmark[(1)]}}}
        & \multicolumn{2}{c}{Coordenadas geográficas}  \\ \cmidrule{4-5}
          &  \multicolumn{1}{l|}{}  & & \multicolumn{1}{|c|}{Latitude} & Longitude 
          \ML
     Pico da Neblina  & Serra do Imeri (AM) & 3\,014,1 & \ang{+00;47;49} & \ang{-66;00;22} \\ \addlinespace
     Pico 31 de Março  & Serra do Imeri (AM)\tmark[(2)] & 2\,992,4 & \ang{+00;48;10} & \ang{-66;00;15} \\ \addlinespace
      Pico da Bandeira  & Serra do Caparaó (MG/ES) & 2\,889,9 & \ang{-20;26;01} & \ang{-41;47;52} \\ \addlinespace
    Pico do Cristal  & Serra do Caparaó (MG) & 2\,798 & \ang{-20;26;37} & \ang{-41;48;42} \\ \addlinespace
    Pico das Agulhas Negras  & Serra do Itatiaia (MG/RJ) & 2\,787 & \ang{-22;22;47} & \ang{-44;39;40} \\ \addlinespace
    Pedra da Mina  & Serra da Mantiqueira (MG/SP) & 2\,770 & \ang{-22;25;38} & \ang{-44;50;33} \\ \addlinespace
    Pico do Calçado  & Serra do Caparaó (ES/MG) & 2\,766 & \ang{-20;27;07} & \ang{-40;50;28} \\ \addlinespace
    Monte Roraima  & Serra do Pacaraima (RR)\tmark[(2)]\tmark[(3)] & 2\,727,3 & \ang{+05;12;05} & \ang{-60;43;39} \\ \addlinespace
    Pico Três Estados  & Serra da Mantiqueira (SP/MG/RJ) & 2\,665 & \ang{-22;24;22} & \ang{-44;48;34} \\ \addlinespace
    Pico do Cadorna  & Serra do Imeri (AM)\tmark[(2)] & 2\,596 & \ang{+00;47;50} & \ang{-66;00;30} \\ \addlinespace
    Pedra Furada  & Serra da Mantiqueira (RJ/MG) & 2\,589 & \ang{-22;21;28} & \ang{-44;43;25} \LL
}

\newpage
\begin{quadroBase}[Código Tabela 15]
{\footnotesize
  \lstinputlisting[firstline=8]{../exemplo/tabelas/tabela-15-modelo.tex} 
}
\end{quadroBase}

	%%
	%%
\end{anexosenv}




\printbibliography[title={Referências}, heading=bibintoc]

%\phantomsection
%\printglossaries

%\cleardoublepage
%\phantomsection
%\printindex

\end{document}