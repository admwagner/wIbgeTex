% Tabela Modelo 8
% Extraída ou adaptada de:
% IBGE, Fundação Instituto Brasileiro de Geografia e Estatística.
% Normas de apresentação tabular. 3. ed. Rio de Janeiro: IBGE, 1993.
%
% Este arquivo é destinado a ser incluído via \input{}

\ctable[
  label = tabIbge8,
  caption = {Superfície total, em números absolutos e relativos, por zona hipsométrica do Brasil -- 1973}, 
  width = \linewidth, center, pos = th, notespar, nosuper
]
{X R{30mm} R{30mm}} %  Definição das colunas
{% notas
  \wIbgeFonte{IBGE, Diretoria de Geociências, Departamento de Cartografia.}
  \wIbgeNota{Dados sujeitos a retificação.}
  \wIbgeNoteFont % Define a fonte para \tnote 
  \tnote[(1)]{ Áreas de reservas ecológicas, conforme resolução n. 04 de 18.09.1985 do Conselho Nacional do Meio Ambiente,} 
}
{ % conteúdo da tabela
  \FL
  \multirow{2}{*}{Zona hipsométrica (m)} & \multicolumn{2}{c}{Superfície total} \\
  &  Absoluta (km\textsuperscript{2}) & Relativa (\%)  \ML

  \hspace{5mm} Total & 8 511 996 & 10 000 \\ \addlinespace
  
  Terras baixas & 3 489 553 & 4 100 \\
  0 a 100 & 2 050 318 & 2 409 \\
  101 a 200 & 1 439 235 & 1 691 \\ \addlinespace
  
  Terras altas & 4 976 176 & 5 846 \\
  201 a 500 & 3 151 646 & 3 703 \\
  501 a 800 & 1 249 906 & 1 468 \\
  801 a 1200 & 574 624 & 675 \\ \addlinespace
  
  Áreas culminantes & 46 267 & 0,54 \\
  1201 a 1800 & 44 767 & 0,52 \\
  1801 a 3014\tmark[(1)] & 1500 & 0,02 \LL
}
